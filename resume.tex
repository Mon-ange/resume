% !TEX program = xelatex

\documentclass{resume}
%\usepackage{zh_CN-Adobefonts_external} % Simplified Chinese Support using external fonts (./fonts/zh_CN-Adobe/)
%\usepackage{zh_CN-Adobefonts_internal} % Simplified Chinese Support using system fonts

\begin{document}
\pagenumbering{gobble} % suppress displaying page number

\name{Zhou Mengting}

\basicInfo{
  \email{98monange@gmail.com} \textperiodcentered\ 
  \phone{(+86) 173-685-95836} \textperiodcentered\ 
  \github[Mon-ange]{https://github.com/Mon-ange}}

\section{\faGraduationCap\ Education}
\datedsubsection{\textbf{Beijing University of Technology(BJUT)}, Beijing, China}{2021 -- Present}
\textit{Master student} in Software Engineering (SE), expected March 2024
\datedsubsection{\textbf{Changzhou University}, JiangSu, China}{2015 -- 2019}
\textit{B.S.} in Computer Science (CS)

\section{\faUsers\ Experience}
\datedsubsection{\textbf{Beijing Roborock Technology Co., Ltd.} Beijing, China}{2021/7/15 -- 2021/11/15}
\role{Software Development Engineer in Test}{Summer Intern}
Brief introduction: Automatic testing and pcap for vacuum on Mihome and Roborock app.
\begin{itemize}
  \item For the products in Internet of things, we should test the products and the apps connected 
  for many times a day. I deployed a automatic testing method to implement them. Thus I came into the
  testing department of roborock, I found that I am the only SDET with coding skills. So I indepently 
  developed a automatic system which can test the functions of mihome and roborock app such as: sweeping, 
  firmware updating, language packet, sheduled clean, remote control and etc. By the way ,this system can 
  automatically grabs the IP packets to local and then analysts can analyze the exceptions. I designed a 
  hierarchical structure to make the code and the methods clean. The skills I use are testng, appium and 
  springboot, and the language I used is Java.
\end{itemize}

% Reference Test
%\datedsubsection{\textbf{Paper Title\cite{zaharia2012resilient}}}{May. 2015}
%An xxx optimized for xxx\cite{verma2015large}
%\begin{itemize}
%  \item main contribution
%\end{itemize}

\section{\faCogs\ Skills}
\begin{itemize}[parsep=0.5ex]
  \item Programming Languages: Java, C/C++, python
  \item Platform: Windows,Linux,MacOS
  \item Languages: English - CET6
  \item Development: Kubernetes, Spring Boot, Mysql, Shell Scripting, Testing, Appium, Testng
  \item hands on experience with AWS
\end{itemize}

\section{\faHeartO\ Honors and Awards}
\datedline{\textit{\nth{2} Prize}, The 8th "Blue Bridge Cup" National Software Competition in C/C++ Programming }{April. 2017}
\datedline{\textit{\nth{3} Prize}, 2017 Jiangsu Collegiate Programming Contest }{May. 2017}
\datedline{\textit{Top 19.78\%}, CCF Certified Software Professional }{March. 2017}
\datedline{\textit{Honorable Mention}, The ACM-ICPC Asia Regional Contest Beijing Site }{November. 2016}
\datedline{\textit{Honorable Mention}, China Collegiate Programming Contest-2016 Hangzhou }{October. 2016}

\section{\faInfo\ Additional Activites}
\begin{itemize}[parsep=0.5ex]
  \item Member, ACM-ICPC Programming Team in Changzhou University
\end{itemize}

%% Reference
%\newpage
%\bibliographystyle{IEEETran}
%\bibliography{mycite}
\end{document}
