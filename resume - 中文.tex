% !TEX program = xelatex

\documentclass{resume}
%\usepackage{zh_CN-Adobefonts_external} % Simplified Chinese Support using external fonts (./fonts/zh_CN-Adobe/)
%\usepackage{zh_CN-Adobefonts_internal} % Simplified Chinese Support using system fonts

\begin{document}
\pagenumbering{gobble} % suppress displaying page number

\name{周梦婷}

\basicInfo{
  \email{98monange@gmail.com} \textperiodcentered\ 
  \phone{(+86) 173-685-95836} \textperiodcentered\ 
  \github[Mon-ange]{https://github.com/Mon-ange}}

\section{\faGraduationCap\ 教育背景}
\datedsubsection{\textbf{北京工业大学(BJUT)}, 北京}{2021 -- 至今}
\textit{在读硕士研究生} 软件工程 (SE), 预计2024年4月毕业
\datedsubsection{\textbf{常州大学}, 江苏,常州}{2015 -- 2019}
\textit{学士} 计算机科学与技术 (CS)

\section{\faUsers\ 实习经历}
\datedsubsection{\textbf{北京石头世纪科技股份有限公司} 北京}{2021/7/15 -- 2021/11/15}
\role{测试开发工程师}{实习}
主要工作: 针对石头和米家app开发对扫地机器人的自动测试系统
\begin{itemize}
  \item 对于智能物联网行业来说,产品和app连接后,产品的功能需要进行反复测试。所以我针对石头app和米家app开发了一个
  测试系统。在我进入石头测试部门的时候,我作为测试开发组的主力,设计架构并实现了层次测试系统。 该系统主要对快连、
  产品指南、下载语音包、清扫、固件升级、定时清扫、查看耗材、触发报错等功能进行自动化测试并在过程中进行抓包存储方便分析
  人士分析IP包。我设计的系统架构清晰,结构解耦,无硬编码;良好运用了面向对象思想,以页面为基础对象,在服务层是一条实
  际用例对应一行业务代码。主要技术栈:Appium、testng、java。
\end{itemize}

% Reference Test
%\datedsubsection{\textbf{Paper Title\cite{zaharia2012resilient}}}{May. 2015}
%An xxx optimized for xxx\cite{verma2015large}
%\begin{itemize}
%  \item main contribution
%\end{itemize}

\section{\faCogs\ IT技能}
\begin{itemize}[parsep=0.5ex]
  \item 编程语言: Java, C/C++, python
  \item 平台: Windows,Linux,MacOS
  \item 语言: English - CET6
  \item 技术栈: Kubernetes, Spring Boot, Mysql, Shell Scripting, Testing, Appium, Testng
  \item 熟悉 AWS
\end{itemize}

\section{\faHeartO\ 获奖情况}
\datedline{第八届蓝桥杯全国软件和信息技术专业人才大赛江苏赛区二等奖(C/C++)  }{2017.04.08}
\datedline{2017江苏省程序设计竞赛(JSCPC)三等奖 }{2017.05.14}
\datedline{CCF-CSP计算机软件能力认证(C/C++)得分210,全国前19.78\%}{2017.03.20}
\datedline{ACM-ICPC国际大学生程序竞赛-亚洲区域赛(北京大学站)荣誉提名 }{2016.11.13}
\datedline{2016年中国大学生程序设计竞赛(杭州)荣誉提名}{2016.10.10}

\section{\faInfo\ 其他活动}
\begin{itemize}[parsep=0.5ex]
  \item ACM国际大学生程序竞赛校队队员,2016年10月 - 2018年10月

\end{itemize}

%% Reference
%\newpage
%\bibliographystyle{IEEETran}
%\bibliography{mycite}
\end{document}
